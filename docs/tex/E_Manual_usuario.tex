\apendice{Documentación de usuario}

\section{Introducción}
En esta sección se detallan los requerimientos de la aplicación, su instalación y despliegue (en el caso de \texttt{UBUMLaaS}) y se acompañan de una serie de indicaciones y consejos para su correcto uso.

De igual manera que en el Manual del Programador cada parte del proyecto, \texttt{IS-SSL} y \texttt{UBUMLaaS}, se describirá por su propio lado, de tal manera que aunque haya aspectos comunes, cada una su propia documentación de usuario.

\section{UBUMLaaS}
\subsection{Requisitos de usuarios}
Los requisitos mínimos para poder hacer uso de \texttt{UBUMLaaS} son:
\begin{itemize}
\item Disponer de una conexión a Internet.
\item Hacer uso de navegador web con soporte a HTML5.
\item Tener habilitado JavaScript en el navegador.
\item Tener una cuenta en la plataforma.
\end{itemize}
\subsection{Instalación}

\subsection{Manual del usuario}



%%%%%%%%%%%%%%%%%%%%%%%%%%%%%%%%%%%%%%%%%%%%%%%%%%%%%%%%%%%%%%%%%%%%%%
\section{IS-SSL}
\subsection{Requisitos de usuarios}
Los requisitos mínimos para poder hacer uso de \texttt{IS-SSL} son:
\begin{itemize}
\item Tener instalado Python 3.7+.
\item Disponer de un editor de textos.
\item Tener instaladas las bibliotecas necesarias para su correcto funcionamiento.
\end{itemize}

\subsection{Instalación}

\subsection{Manual del usuario}


