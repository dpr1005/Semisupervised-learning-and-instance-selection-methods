\capitulo{3}{Conceptos teóricos}

El proyecto tiene una relación directa con la minería de datos y los conceptos que lo rodean. 

\section{Minería de datos}

Según IBM \cite{IBM-WhatisDataMining}, podemos definir la minería de datos, o descubrimiento de conocimiento
en los datos \textit{knowledge Discovery in Databases}, como el proceso de descubrir patrones y otra
información a partir de grandes conjuntos de datos. 

Las técnicas de minería de datos principales se pueden dividir en función de sus propósitos principales.
\begin{enumerate}
    \item Descripción del conjunto de datos objetivo.
    \item Predicción de resultados mediante el uso de algoritmos de aprendizaje automático.
\end{enumerate}

\subsection{Proceso de minería de datos}
El proceso seguido en la minería de datos es muy directo. Comienza con la recogida de los datos que van a ser
tratados. y finaliza con la visualización de la información extraída de éstos. 
Los científicos de datos describen los datos a través de sus observaciones de patrones, asociaciones y correlaciones. A su vez se pueden clasificar y agrupar los datos utilizando métodos de clasificación y regresión.

Uno de los marcos de referencia más importantes en el proceso de minado de datos es CRISP-DM, \textit{Cross Industry Standard Process for Data Mining}. Desarrollado por un consorcio de empresas involucradas en la minería de datos. \cite{Chapman2000CRISPDM1S}

\imagen{../img/CRISP-DM}{Enfoque CRISP de la minería de datos.}

En \cite{KOTU201517} se divide el proceso de la minería de datos en 5 etapas o pasos principales: establecimiento de los objetivos y comprensión del problema, recopilación y preparación de los datos, desarrollo del modelo, aplicación del modelo y la evaluación de los resultados y despliegue en producción.

\begin{enumerate}
   \item \textbf{Establecer los objetivos y comprensión del problema.}
    La primera etapa puede resultar la más complicada del proceso. Todas las partes interesadas deben de estar presentes y de acuerdo en la definición del problema que se va tratar, esto incluye tanto a los científicos de datos como las terceras partes involucradas o interesadas. 
    Este procedimiento ayuda a la formulación de las preguntas de los datos y los parámetros a utilizar en el proyecto. Si se trata de un proyecto empresarial, se debe hacer un estudio o investigación adicional para comprender el contexto de la empresa.
    \item \textbf{Preparación de los datos.}
    Con el alcance del problema definido ya se puede comenzar a identificar qué conjunto de datos será el más efectivo o representativo con el fin de comenzar a dar respuesta a las preguntas formuladas en el proceso anterior.
    
    Una vez se tienen todos los datos recogidos comienza el proceso de pre-procesado de los mismos. Este proceso se basa en la limpieza de los datos con el fin de eliminar cualquier posible ruido, entendiéndose por ruido los datos duplicados, los valores perdidos y aquellos atípicos; aquellos que puedan causar problemas a la resolución del problema o generen incertidumbre.
    En determinados conjuntos de datos se puede hacer una reducción de dimensiones. Consiste en la reducción del número de dimensiones que poseen las instancias recogidas, con el fin de eliminar aquellas que no sean realmente representativas o significativas, este proceso reduce la complejidad de los cálculos posteriores. Por contrapartida hay que conocer cuáles serán los predictores con mayor relevancia en el problema para garantizar una precisión "óptima" del modelo.
    \item \textbf{Desarrollo del modelo.}
    Según \cite{KOTU201517} el modelo es la representación abstracta de los datos y sus relaciones en un conjunto de datos concreto. Actualmente existen cientos de algoritmos que se pueden utilizar, habitualmente proceden de campos como la ciencia de datos, \textit{machine learning}, o la estadística.
    Se debe tener el conocimiento suficiente para entender como funciona el algoritmo para poder configurar correctamente los parámetros que este va a utilizar en base a los datos y el problema de negocio que estamos resolviendo. 
    
    Los modelos en función de como resuelvan el problema que se les presenta se pueden clasificar en:
    \begin{enumerate}
        \item Regresión.
        \item Análisis de asociación.
        \item \textit{Clustering.}
        \item Detección de anomalías.
    \end{enumerate}
    
    El modelo debe ser creado con especial cuidado para evitar el \textit{overfitting}, i.e. el modelo memoriza el conjunto de entrenamiento y no tendrá un rendimiento correcto una vez desplegado en producción. Se desea que el modelo sea lo más generalizado posible de cara a \textit{aprender} de los datos del conjunto de entrenamiento.

    \item \textbf{Aplicación del modelo.}
	El momento de la aplicación del modelo es cuando de verdad se comprueba si realmente el modelo está listo para pasar al siguiente punto, en otras palabras, si es apto para ser desplegado en producción. 
	Para ello se tienen en cuenta métricas como la calidad del modelo ante el problema, su tiempo de respuesta, etc.
    
	\item \textbf{Evaluación de los resultados y despliegue en producción.}
	Una vez que el modelo se encuentra listo es desplegado en producción. Es habitual que los parámetros con los que el modelo fue entrenado con el paso del tiempo dejen de ser los más interesantes, pudiendo ser comprobado el error proporcionado por el modelo con los datos de prueba. Cuando ese error sea excesivo o fuera de un margen dado se deberá de volver a entrenar el modelo, comprobar, y desplegar. 
	De esta forma se puede comprobar como el ciclo de vida del modelo es circular.
\end{enumerate}

El proceso aplicado en la minería de datos proporciona un marco de trabajo mediante el cual se permite extraer información aparentemente no trivial de grandes conjuntos de datos. 
Es un campo de aprendizaje constante, tanto el aplicar los conocimientos del analista para reducir las dimensiones del conjunto de datos, como una vez que se ha entrenado el modelo y puesto en producción aprender los puntos fuertes de este y el porqué de éstos.\cite{Chapman2000CRISPDM1S}

\subsection{Técnicas utilizadas en la minería de datos}
Como se ha comentado anteriormente, uno de los mayores problemas de cara al minado de datos es la dimensionalidad que en muchas ocasiones tienen estos. Es por ello que se aplican técnicas o algoritmos que faciliten la extracción de la información útil. Algunos de ellos son:
\begin{enumerate}
	\item \textbf{Reglas de asociación.} Dado un conjunto de datos concreto, consiste en la aplicación de reglas para encontrar relaciones entre las variables.
	\item \textbf{Redes neuronales.} Principalmente utilizadas en \textit{deep learning}, simulan la interconectividad propia del cerebro humano utilizando capas de nodos. Cada nodo está compuesto por \textit{[\x_n\]} entradas, \textit{[\w_n\]} pesos y un sesgo o umbral, el cual al ser superado activa la neurona, pasando los datos del nodo a la siguiente neurona. Habitualmente con una única iteración sobre la red neuronal, esta es capaz de obtener una solución medianamente buena de un conjunto de datos de tamaño considerable. \cite{CRAVEN1997211}
	\item \textbf{Árboles de decisión.} Mediante el uso de métodos de clasificación y regresión se clasifican o predicen potenciales resultados en función de un conjunto de decisiones. Utiliza una visualización en forma de árbol para representar los posibles resultados de estas decisiones.
	\item \textbf{k-vecinos más cercanos. KNN (\textit{k-nearest neighbors})} Método no parmétrico de clasificación, sencillo pero eficaz.\cite{hand2007principles} Para clasificar un conjunto de datos \textit{T}, se recuperan sus \textit{k} vecinos más cercanos, que forman una vecindad de \textit{T}. Se suele utilizar la votación por mayoría entre los registros de datos de la vecindad para decidir la clasificación de \textit{T} con o sin consideración de la ponderación basada en la distancia.\cite{guo2003knn}
\end{enumerate}