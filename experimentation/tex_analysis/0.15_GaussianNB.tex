\documentclass[a4paper,10pt]{article}
\usepackage{graphicx}
\usepackage{lscape}
\title{Output tables for 1xN statistical comparisons.}
\author{}
\date{\today}
\begin{document}
\begin{landscape}
\pagestyle{empty}
\maketitle
\thispagestyle{empty}

\section{Average rankings of Friedman test}


Average ranks obtained by each method in the Friedman test.

\begin{table}[!htp]
\centering
\begin{tabular}{|c|c|}\hline
Algorithm&Ranking\\\hline
GaussianNB:ENANE&2.5\\GaussianNB:ENN&2.5556\\GaussianNB:LSSm&2.5556\\GaussianNB:base&2.3889\\\hline\end{tabular}
\caption{Average Rankings of the algorithms (Friedman)}
\end{table}

Friedman statistic (distributed according to chi-square with 3 degrees of freedom): 0.2. \newline P-value computed by Friedman Test: 0.977589.\newline


\newpage

\section{Post hoc comparison (Friedman)}


P-values obtained in by applying post hoc methods over the results of Friedman procedure.

\begin{table}[!htp]
\centering\footnotesize
\begin{tabular}{ccccc}
$i$&algorithm&$z=(R_0 - R_i)/SE$&$p$&Hochberg \\
\hline3&GaussianNB:ENN&0.387298&0.698535&0.016667\\2&GaussianNB:LSSm&0.387298&0.698535&0.025\\1&GaussianNB:ENANE&0.258199&0.796253&0.05\\\hline
\end{tabular}
\caption{Post Hoc comparison Table for $\alpha=0.05$ (FRIEDMAN)}
\end{table}
\newpage

\section{Adjusted P-Values (Friedman)}


Adjusted P-values obtained through the application of the post hoc methods (Friedman).

\begin{table}[!htp]
\centering\small
\begin{tabular}{cccc}
i&algorithm&unadjusted $p$&$p_{Hochberg}$\\
\hline1&GaussianNB:ENN&0.698535&0.796253\\2&GaussianNB:LSSm&0.698535&0.796253\\3&GaussianNB:ENANE&0.796253&0.796253\\\hline
\end{tabular}
\caption{Adjusted $p$-values (FRIEDMAN) (I)}
\end{table}
\begin{table}[!htp]
\centering\small
\begin{tabular}{ccc}
i&algorithm&unadjusted $p$\\
\hline1&GaussianNB:ENN&0.698535\\2&GaussianNB:LSSm&0.698535\\3&GaussianNB:ENANE&0.796253\\\hline
\end{tabular}
\caption{Adjusted $p$-values (FRIEDMAN) (II)}
\end{table}

\newpage
\end{landscape}\end{document}